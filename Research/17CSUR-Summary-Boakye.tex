\documentclass[dvips,12pt]{article}

% Any percent sign marks a comment to the end of the line

% Every latex document starts with a documentclass declaration like this
% The option dvips allows for graphics, 12pt is the font size, and article
%   is the style

\usepackage[pdftex]{graphicx}
\usepackage{url}
\usepackage{amsbsy}
\usepackage{amsmath}
\usepackage{array}
\newcolumntype{L}{>{\centering\arraybackslash}m{2cm}}


% These are additional packages for "pdflatex", graphics, and to include
% hyperlinks inside a document.
\setlength{\oddsidemargin}{0.25in}
\setlength{\textwidth}{6.5in}
\setlength{\topmargin}{0in}
\setlength{\textheight}{8.5in}

% These force using more of the margins that is the default style

\begin{document}

% Everything after this becomes content
% Replace the text between curly brackets with your own

\title{Control Flow Integrity: Security Precision and Performance - A Summary}
\author{Boakye Dankwa}
\date{\today}

% You can leave out "date" and it will be added automatically for today
% You can change the "\today" date to any text you like


\maketitle

% This command causes the title to be created in the document

%\section{Abstract}
Control-Flow Integrity (CFI) shows promise to defeat control flow modification attacks such as remote code injection, Return-Oriented Programing (ROP) and code-reuse. The technique has been researched and improved upon by researchers and been integrated into products such as LLVM. Current CFI evaluations usually use the SPEC2006 benchmark to measure performance and Average Indirect-target Reduction (AIR) to measure security assurance. The authors in   \cite{DBLP:journals/corr/BurowCBPNLF16} propose a novel 




 
%\begin{thebibliography}{99}

%\bibitem{gonzalez2012} Jonay I. Gonz\'{a}lez Hern\'{a}ndez, 
%Pilar Ruiz-Lapuente,	
%Hugo M. Tabernero,	
%David Montes,	
%Ramon Canal,	
%Javier M\'{e}ndez	
%and Luigi R. Bedin,
%{No surviving evolved companions of the progenitor of SN1006},
%Nature, {\bf 489}, 533-536 (2012).

%\end{thebibliography}

\bibliographystyle{plain}
\bibliography{References}



\end{document}
